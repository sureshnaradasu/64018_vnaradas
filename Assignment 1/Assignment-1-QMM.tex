% Options for packages loaded elsewhere
\PassOptionsToPackage{unicode}{hyperref}
\PassOptionsToPackage{hyphens}{url}
%
\documentclass[
]{article}
\usepackage{amsmath,amssymb}
\usepackage{iftex}
\ifPDFTeX
  \usepackage[T1]{fontenc}
  \usepackage[utf8]{inputenc}
  \usepackage{textcomp} % provide euro and other symbols
\else % if luatex or xetex
  \usepackage{unicode-math} % this also loads fontspec
  \defaultfontfeatures{Scale=MatchLowercase}
  \defaultfontfeatures[\rmfamily]{Ligatures=TeX,Scale=1}
\fi
\usepackage{lmodern}
\ifPDFTeX\else
  % xetex/luatex font selection
\fi
% Use upquote if available, for straight quotes in verbatim environments
\IfFileExists{upquote.sty}{\usepackage{upquote}}{}
\IfFileExists{microtype.sty}{% use microtype if available
  \usepackage[]{microtype}
  \UseMicrotypeSet[protrusion]{basicmath} % disable protrusion for tt fonts
}{}
\makeatletter
\@ifundefined{KOMAClassName}{% if non-KOMA class
  \IfFileExists{parskip.sty}{%
    \usepackage{parskip}
  }{% else
    \setlength{\parindent}{0pt}
    \setlength{\parskip}{6pt plus 2pt minus 1pt}}
}{% if KOMA class
  \KOMAoptions{parskip=half}}
\makeatother
\usepackage{xcolor}
\usepackage[margin=1in]{geometry}
\usepackage{color}
\usepackage{fancyvrb}
\newcommand{\VerbBar}{|}
\newcommand{\VERB}{\Verb[commandchars=\\\{\}]}
\DefineVerbatimEnvironment{Highlighting}{Verbatim}{commandchars=\\\{\}}
% Add ',fontsize=\small' for more characters per line
\usepackage{framed}
\definecolor{shadecolor}{RGB}{248,248,248}
\newenvironment{Shaded}{\begin{snugshade}}{\end{snugshade}}
\newcommand{\AlertTok}[1]{\textcolor[rgb]{0.94,0.16,0.16}{#1}}
\newcommand{\AnnotationTok}[1]{\textcolor[rgb]{0.56,0.35,0.01}{\textbf{\textit{#1}}}}
\newcommand{\AttributeTok}[1]{\textcolor[rgb]{0.13,0.29,0.53}{#1}}
\newcommand{\BaseNTok}[1]{\textcolor[rgb]{0.00,0.00,0.81}{#1}}
\newcommand{\BuiltInTok}[1]{#1}
\newcommand{\CharTok}[1]{\textcolor[rgb]{0.31,0.60,0.02}{#1}}
\newcommand{\CommentTok}[1]{\textcolor[rgb]{0.56,0.35,0.01}{\textit{#1}}}
\newcommand{\CommentVarTok}[1]{\textcolor[rgb]{0.56,0.35,0.01}{\textbf{\textit{#1}}}}
\newcommand{\ConstantTok}[1]{\textcolor[rgb]{0.56,0.35,0.01}{#1}}
\newcommand{\ControlFlowTok}[1]{\textcolor[rgb]{0.13,0.29,0.53}{\textbf{#1}}}
\newcommand{\DataTypeTok}[1]{\textcolor[rgb]{0.13,0.29,0.53}{#1}}
\newcommand{\DecValTok}[1]{\textcolor[rgb]{0.00,0.00,0.81}{#1}}
\newcommand{\DocumentationTok}[1]{\textcolor[rgb]{0.56,0.35,0.01}{\textbf{\textit{#1}}}}
\newcommand{\ErrorTok}[1]{\textcolor[rgb]{0.64,0.00,0.00}{\textbf{#1}}}
\newcommand{\ExtensionTok}[1]{#1}
\newcommand{\FloatTok}[1]{\textcolor[rgb]{0.00,0.00,0.81}{#1}}
\newcommand{\FunctionTok}[1]{\textcolor[rgb]{0.13,0.29,0.53}{\textbf{#1}}}
\newcommand{\ImportTok}[1]{#1}
\newcommand{\InformationTok}[1]{\textcolor[rgb]{0.56,0.35,0.01}{\textbf{\textit{#1}}}}
\newcommand{\KeywordTok}[1]{\textcolor[rgb]{0.13,0.29,0.53}{\textbf{#1}}}
\newcommand{\NormalTok}[1]{#1}
\newcommand{\OperatorTok}[1]{\textcolor[rgb]{0.81,0.36,0.00}{\textbf{#1}}}
\newcommand{\OtherTok}[1]{\textcolor[rgb]{0.56,0.35,0.01}{#1}}
\newcommand{\PreprocessorTok}[1]{\textcolor[rgb]{0.56,0.35,0.01}{\textit{#1}}}
\newcommand{\RegionMarkerTok}[1]{#1}
\newcommand{\SpecialCharTok}[1]{\textcolor[rgb]{0.81,0.36,0.00}{\textbf{#1}}}
\newcommand{\SpecialStringTok}[1]{\textcolor[rgb]{0.31,0.60,0.02}{#1}}
\newcommand{\StringTok}[1]{\textcolor[rgb]{0.31,0.60,0.02}{#1}}
\newcommand{\VariableTok}[1]{\textcolor[rgb]{0.00,0.00,0.00}{#1}}
\newcommand{\VerbatimStringTok}[1]{\textcolor[rgb]{0.31,0.60,0.02}{#1}}
\newcommand{\WarningTok}[1]{\textcolor[rgb]{0.56,0.35,0.01}{\textbf{\textit{#1}}}}
\usepackage{graphicx}
\makeatletter
\def\maxwidth{\ifdim\Gin@nat@width>\linewidth\linewidth\else\Gin@nat@width\fi}
\def\maxheight{\ifdim\Gin@nat@height>\textheight\textheight\else\Gin@nat@height\fi}
\makeatother
% Scale images if necessary, so that they will not overflow the page
% margins by default, and it is still possible to overwrite the defaults
% using explicit options in \includegraphics[width, height, ...]{}
\setkeys{Gin}{width=\maxwidth,height=\maxheight,keepaspectratio}
% Set default figure placement to htbp
\makeatletter
\def\fps@figure{htbp}
\makeatother
\setlength{\emergencystretch}{3em} % prevent overfull lines
\providecommand{\tightlist}{%
  \setlength{\itemsep}{0pt}\setlength{\parskip}{0pt}}
\setcounter{secnumdepth}{-\maxdimen} % remove section numbering
\ifLuaTeX
  \usepackage{selnolig}  % disable illegal ligatures
\fi
\IfFileExists{bookmark.sty}{\usepackage{bookmark}}{\usepackage{hyperref}}
\IfFileExists{xurl.sty}{\usepackage{xurl}}{} % add URL line breaks if available
\urlstyle{same}
\hypersetup{
  pdftitle={QMM assignment 1},
  pdfauthor={Venkata Suresh Naradasu},
  hidelinks,
  pdfcreator={LaTeX via pandoc}}

\title{QMM assignment 1}
\author{Venkata Suresh Naradasu}
\date{2023-09-23}

\begin{document}
\maketitle

\begin{center}\rule{0.5\linewidth}{0.5pt}\end{center}

This Notebook contains the code for Assignment 1

\hypertarget{summary}{%
\section{Summary}\label{summary}}

\begin{enumerate}
\def\labelenumi{\arabic{enumi}.}
\tightlist
\item
  Maximum revenue = \$1780 by making 40 Artisanal Truffles, 12
  Handcrafted Chocolate Nuggets, 4 Premium Gourmet Chocolate Bars per
  day
\end{enumerate}

Each ingrediant (cacao butter, honey, cream) constraints are binding.

\begin{enumerate}
\def\labelenumi{\arabic{enumi}.}
\setcounter{enumi}{1}
\tightlist
\item
  Cacao Butter constraint: Shadow price = \$2, Range of feasibility =
  47.5 to 51.6 cups. Honey constraint: Shadow price = \$30, Range of
  feasibility = 30 to 52 cups. Cream constraint: Shadow price = \$6,
  Range of feasibility = 29 to 50 cups.
\end{enumerate}

3.If the local store increases the daily order to 25 pounds of chocolate
nuggets. Maximum Revenue= \$1558 \& Each Product that Francesco should
produce is 26.6 Artisanal Truffles 25 Handcrafted Chocolate 0 Premium
Gourmet Chocolate Bars

\begin{center}\rule{0.5\linewidth}{0.5pt}\end{center}

Load lpSolveAPI

\begin{Shaded}
\begin{Highlighting}[]
\FunctionTok{library}\NormalTok{(lpSolveAPI)}
\end{Highlighting}
\end{Shaded}

\begin{center}\rule{0.5\linewidth}{0.5pt}\end{center}

\textbf{Problem Statement: }

A renowned chocolatier, Francesco Schröeder, makes three kinds of
chocolate confectionery: artisanal truffles, handcrafted chocolate
nuggets, and premium gourmet chocolate bars. He uses the highest quality
of cacao butter, dairy cream, and honey as the main ingredients.
Francesco makes his chocolates each morning, and they are usually sold
out by the early afternoon. For a pound of artisanal truffles, Francesco
uses 1 cup of cacao butter, 1 cup of honey, and 1/2 cup of cream. The
handcrafted nuggets are milk chocolate and take 1/2 cup of cacao, 2/3
cup of honey, and 2/3 cup of cream for each pound. Each pound of the
chocolate bars uses 1 cup of cacao butter, 1/2 cup of honey, and 1/2 cup
of cream. One pound of truffles, nuggets, and chocolate bars can be
purchased for \$35, \$25, and \$20, respectively. A local store places a
daily order of 10 pounds of chocolate nuggets, which means that
Francesco needs to make at least 10 pounds of the chocolate nuggets each
day. Before sunrise each day, Francesco receives a delivery of 50 cups
of cacao butter, 50 cups of honey, and 30 cups of dairy cream.

1.) Formulate and solve the LP model that maximizes revenue given the
constraints. How much of each chocolate product should Francesco make
each morning? What is the maximum daily revenue that he can make? 2.)
Report the shadow price and the range of feasibility of each binding
constraint. 3.) If the local store increases the daily order to 25
pounds of chocolate nuggets, how much of each product should Francesco
make?

\begin{center}\rule{0.5\linewidth}{0.5pt}\end{center}

We will solve this problem with two approaches by directly encoding the
variables and coefficients.

\begin{center}\rule{0.5\linewidth}{0.5pt}\end{center}

We define the following:

\begin{itemize}
\tightlist
\item
  Decision Variables: Let \emph{A} be the number of Pounds of Artisanal
  Truffles produced, and \emph{H} be the number of Pounds of Handcrafted
  Chocolate produced and \emph{P} be the number of Pounds of Premium
  Gourmet Chocolate Bars produced
\item
  The Objective is to \emph{Max 35A + 25H + 20P}. The constraints are

  \begin{itemize}
  \tightlist
  \item
    \emph{cacao butter: 1A + 0.5H + 1P \textless= 50;}
  \item
    \emph{Honey: 1A + 2/3H + 0.5P \textless= 50;}
  \item
    \emph{Cream: 0.5A + 2/3H + 0.5P \textless= 25;}
  \item
    \emph{Local\_Store\_order: 1H \textless= 10;}
  \item
    Non-negativity constraints
  \end{itemize}
\end{itemize}

We now solve the above LP problem

\begin{Shaded}
\begin{Highlighting}[]
\FunctionTok{solve}\NormalTok{(lprec)}
\end{Highlighting}
\end{Shaded}

\begin{verbatim}
## [1] 0
\end{verbatim}

0 implies that the model has been successfully solved. Now we need to
find the values of the decision variables and maximum revenue

\begin{Shaded}
\begin{Highlighting}[]
\FunctionTok{get.objective}\NormalTok{(lprec)}
\end{Highlighting}
\end{Shaded}

\begin{verbatim}
## [1] 1780
\end{verbatim}

\begin{Shaded}
\begin{Highlighting}[]
\FunctionTok{get.variables}\NormalTok{(lprec)}
\end{Highlighting}
\end{Shaded}

\begin{verbatim}
## [1] 40 12  4
\end{verbatim}

\begin{Shaded}
\begin{Highlighting}[]
\NormalTok{varV }\OtherTok{\textless{}{-}} \FunctionTok{get.variables}\NormalTok{(lprec)}
\end{Highlighting}
\end{Shaded}

The solution shows that the revenue is 1780, with the first variable
value equal to 40, and the second variable value equal to 12 , and the
second variable value equal to 4 .

Now we read the lp file that we have written using ``write.lp(lprec,
filename =''Q1.lp'', type = ``lp'')'' and copy that to the variable x
for further using it to get sensitivity values

\begin{Shaded}
\begin{Highlighting}[]
\CommentTok{\#setwd("C:\textbackslash{}\textbackslash{}Users\textbackslash{}\textbackslash{}sures\textbackslash{}\textbackslash{}Downloads")}
\NormalTok{x }\OtherTok{\textless{}{-}} \FunctionTok{read.lp}\NormalTok{(}\StringTok{"Q1.lp"}\NormalTok{)}
\NormalTok{x}
\end{Highlighting}
\end{Shaded}

\begin{verbatim}
## Model name: 
##                               Artisanal_Truffles  handcrafted_chocolate_nuggets                 chocolate_bars        
## Maximize                                      35                             25                             20        
## cacao_butter                                   1                            0.5                              1  <=  50
## honey                                          1                 0.666666666667                            0.5  <=  50
## cream                                        0.5                 0.666666666667                            0.5  <=  30
## Local_Store_order                              0                              1                              0  >=  10
## Kind                                         Std                            Std                            Std        
## Type                                        Real                           Real                           Real        
## Upper                                        Inf                            Inf                            Inf        
## Lower                                          0                              0                              0
\end{verbatim}

Now we solve the model with copied variable x and get objective maximum
values and decision variable again. Since without solving the model with
variable ``x'' we cannot get sensitivity values.

\begin{Shaded}
\begin{Highlighting}[]
\FunctionTok{solve}\NormalTok{(x)}
\end{Highlighting}
\end{Shaded}

\begin{verbatim}
## [1] 0
\end{verbatim}

\begin{Shaded}
\begin{Highlighting}[]
\FunctionTok{get.objective}\NormalTok{(x)}
\end{Highlighting}
\end{Shaded}

\begin{verbatim}
## [1] 1780
\end{verbatim}

\begin{Shaded}
\begin{Highlighting}[]
\FunctionTok{get.variables}\NormalTok{(x)}
\end{Highlighting}
\end{Shaded}

\begin{verbatim}
## [1] 40 12  4
\end{verbatim}

\begin{Shaded}
\begin{Highlighting}[]
\FunctionTok{get.constraints}\NormalTok{(x)}
\end{Highlighting}
\end{Shaded}

\begin{verbatim}
## [1] 50 50 30 12
\end{verbatim}

We now get the sensitivity values of objective function and dual problem
as well

\begin{Shaded}
\begin{Highlighting}[]
\FunctionTok{get.sensitivity.obj}\NormalTok{(x)}
\end{Highlighting}
\end{Shaded}

\begin{verbatim}
## $objfrom
## [1] 20.00 22.50 18.75
## 
## $objtill
## [1] 38.00000 26.66667 35.00000
\end{verbatim}

\begin{Shaded}
\begin{Highlighting}[]
\FunctionTok{get.sensitivity.rhs}\NormalTok{(x)}
\end{Highlighting}
\end{Shaded}

\begin{verbatim}
## $duals
## [1]  2 30  6  0  0  0  0
## 
## $dualsfrom
## [1]  4.750000e+01  3.000000e+01  2.916667e+01 -1.000000e+30 -1.000000e+30
## [6] -1.000000e+30 -1.000000e+30
## 
## $dualstill
## [1] 5.166667e+01 5.200000e+01 5.000000e+01 1.000000e+30 1.000000e+30
## [6] 1.000000e+30 1.000000e+30
\end{verbatim}

\end{document}
